\documentclass[10pt,a4paper,notitlepage]{article}
\usepackage[utf8]{inputenc}
\usepackage{amsmath}
\usepackage{amsfonts}
\usepackage{amssymb}
\usepackage{amsthm}
\usepackage{graphicx}
\usepackage[left=2cm,right=2cm,top=2cm,bottom=2cm]{geometry}
\usepackage{anyfontsize}
\usepackage{parskip}

%Theorem environment for problem
\theoremstyle{definition}
\newtheorem{prob}{Problem}

%Symbol for \divides
\newcommand{\divides}{\ensuremath{\nobreak \, \vert\, \nobreak}}

\makeatletter
%Harmonise parskip and theorem environments
\def\thm@space@setup
{
    \thm@preskip=\parskip
    \thm@postskip=0pt
}

%Different title style for Samasya
\def\@maketitle{%
    \newpage
    \null
    \vskip 2em%
    \begin{center}%
        \let \footnote \thanks
        {\LARGE \@title \par}%
        \vskip 1.5em%
        %        {\large
        %            \lineskip .5em%
        %            \begin{tabular}[t]{c}%
        %                \@author
        %            \end{tabular}\par}
        %        \vskip 1em%
        {\large \@date}%
    \end{center}
    %    \par
    %    \vskip 1.5em}
}
\makeatother

\title{\textrm{\textbf{\fontsize{30}{40}\selectfont Samasya}}}
\date{%You can put a date here. \today works fine as well.
    }

\begin{document}

\maketitle

Samasya is a mathematics discussion and problem solving club.
We discuss a variety of mathematical topics and solve problems as well.
We encourage participants to have a look at these problems %\footnote{The problems are not necessarily in the order of difficulty}
 before the meeting.
Discussion, however, will not be limited to these problems.
Participants can bring their own problems or mathematical ideas they wish to discuss.\\
\hrule

\textbf{Date: Friday, 4\textsuperscript{th} September}%Date for the meeting
\\
\textbf{Time: 9 p.m.}%Time for the meeting
\\
\textbf{Venue: OPB LAN Room}%Venue for the meeting
\\
\hrule

%Insert problems in the environment prob.
%\begin{prob}
%insert problem statement here
%\end{prob}

\begin{prob}
An algebraic number is a number which is the root of a polynomial with integer coefficients. For example, all rational number are algebraic because they are the root of polynomials of the form $mx-n$, where $m$ and $n$ are integers and $m \neq 0$. Some irrational numbers are also algebraic, e.g. $\sqrt{2}$ is algebraic because it is the root of $x^2-2$. Show that there exists a bijection between the set of algebraic numbers in $\mathbb{R}$ and the natural numbers $\mathbb{N}$. By doing so, prove that there exist some real numbers which are not algebraic. Such numbers are called transcendental.
\end{prob}

\begin{prob}
Let $\mathbb{C}^{\times}$ denote the complex numbers excluding $0$. Let $f$ be a function from $\mathbb{C}^{\times}$ to $\mathbb{C}^{\times}$ such that $f(uf(v))= \frac{f(u)}{v}$ for all $u$ and $v$ in $\mathbb{C}^{\times}$. Show that this function is bijective and $f(ab)=f(a)f(b)$ for all $a$ and $b$ in $\mathbb{C}^{\times}$. Construct an example of such a function, if possible, otherwise prove no such function exists.

Does such a function still exist when $\mathbb{C}^{\times}$ is replaced with $\mathbb{Q}^{+}$, the positive rationals?
\end{prob}

\begin{prob}
Show that one cannot tile a $28 \times 17$ rectangle with rectangles of dimension $4 \times 7$. What about the more general case of tiling an $m \times n$ rectangle with rectangles of dimension $a \times b$? Find the necessary and sufficient conditions for a tiling to be possible.
\end{prob}

\begin{prob}
Let $a$ be any real number. Show that for all values of $\epsilon>0$, there exist positive integers $m$ and $n$ such that $|ma-n|<\epsilon$.
\end{prob}

\begin{prob}
Prove that for any finite sequence of digits, there exists a power of two that starts with these digits.
\end{prob}

\begin{prob}
For any natural number $n$, define $E(n)$ to be the highest exponent of a prime that divides it. For example, $E(27)=3$ and $E(18)=2$. Show that the following limit exists:
$$\lim_{N\to\infty}\frac{1}{N}\sum_{n=2}^{N}E(n)$$
\end{prob}

\end{document}
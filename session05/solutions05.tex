\documentclass[10pt,a4paper,notitlepage]{article}
\usepackage[utf8]{inputenc}
\usepackage{amsmath}
\usepackage{amsfonts}
\usepackage{amssymb}
\usepackage{amsthm}
\usepackage{graphicx}
\usepackage[left=2cm,right=2cm,top=2cm,bottom=2cm]{geometry}
\usepackage{anyfontsize}
\usepackage{parskip}

%Theorem environment for problem
\theoremstyle{definition}
\newtheorem{sol}{Solution}

%Symbol for \divides
\newcommand{\divides}{\ensuremath{\nobreak \, \vert\, \nobreak}}

\makeatletter
%Harmonise parskip and theorem environments
\def\thm@space@setup
{
    \thm@preskip=\parskip
    \thm@postskip=0pt
}

%Different title style for Samasya
\def\@maketitle{%
    \newpage
    \null
    \vskip 2em%
    \begin{center}%
        \let \footnote \thanks
        {\LARGE \@title \par}%
        \vskip 1.5em%
        %        {\large
        %            \lineskip .5em%
        %            \begin{tabular}[t]{c}%
        %                \@author
        %            \end{tabular}\par}
        %        \vskip 1em%
        {\large \@date}%
    \end{center}
    %    \par
    %    \vskip 1.5em}
}
\makeatother

\title{\textrm{\textbf{\fontsize{30}{40}\selectfont Samasya Solutions}}}
\date{%You can put a date here. \today works fine as well.
    }

\begin{document}

\maketitle

These are sketches of solutions to the problems discussed at Samasya on the date mentioned below. Some problems are without a solution; that probably means we were unable to solve them. If you think you have a solution for one of those unsolved problems, feel free to contact the site maintainer. 
\\
\hrule

\textbf{Date: 4\textsuperscript{th} September, 2015}%Date for the meeting
\\
\hrule

%Insert problems in the environment prob.
%\begin{prob}
%insert problem statement here
%\end{prob}

\begin{sol}[Closed]
Countability argument: algebraic numbers are countable, real numbers are not, ergo, there exist uncountably many transcendental numbers.
\end{sol}

\begin{sol}[Semi-open]
Showing the function is an isomorphism is standard symbol chasing. For $\mathbb{Q}^+$, construct the isomorphism by mapping the $(2k+1)$\textsuperscript{th} prime to the reciprocal of the $2k$\textsuperscript{th} prime and map the $2k$\textsuperscript{th} prime to the $(2k+1)$\textsuperscript{th} prime.

For $\mathbb{C}^\times$, one would try $f(x)= x^i$, but that does not work out. However, no proof was found for the fact that no such function exists.
\end{sol}

\begin{sol}[Open]
This problem was not discussed.
\end{sol}

\begin{sol}[Closed]
It's sufficient to prove for an irrational $a$. Consider the sequence $\{na\}$, where $n \in \mathbb{N}$ and $\{\cdot\}$ is the greatest integer function. One needs to show that this sequence is dense in $(0,1)$. That is shown using the pigeon hole principle. The statement of the problems follows through after this.
\end{sol}

\begin{sol}[Closed]
The problem essentially asks for integer solutions in $y$ and $z$ for the following inequality:
$$ x \cdot 10^y \leq 2^z < (x+1) \cdot 10^y$$
Taking $\log_2$ throughout, one gets an inequation of the form $n-ma < \epsilon$, which has an integer solution, by problem 4.
\end{sol}

\begin{sol}[Closed]
(To fill in the details) Riemann-Zeta
\end{sol}

\end{document}

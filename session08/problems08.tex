\documentclass[10pt,a4paper,notitlepage]{article}
\usepackage[utf8]{inputenc}
\usepackage{amsmath}
\usepackage{amsfonts}
\usepackage{amssymb}
\usepackage{amsthm}
\usepackage{graphicx}
\usepackage[left=2cm,right=2cm,top=2cm,bottom=2cm]{geometry}
\usepackage{anyfontsize}
\usepackage{parskip}
\usepackage{hyperref}

%Theorem environment for problem
\theoremstyle{definition}
\newtheorem{prob}{Problem}

%Symbol for \divides
\newcommand{\divides}{\ensuremath{\nobreak \, \vert\, \nobreak}}

\makeatletter
%Harmonise parskip and theorem environments
\def\thm@space@setup
{
    \thm@preskip=\parskip
    \thm@postskip=0pt
}

%Different title style for Samasya
\def\@maketitle{%
    \newpage
    \null
    \vskip 2em%
    \begin{center}%
        \let \footnote \thanks
        {\LARGE \@title \par}%
        \vskip 1.5em%
        %        {\large
        %            \lineskip .5em%
        %            \begin{tabular}[t]{c}%
        %                \@author
        %            \end{tabular}\par}
        %        \vskip 1em%
        {\large \@date}%
    \end{center}
    %    \par
    %    \vskip 1.5em}
}
\makeatother

\title{\textrm{\textbf{\fontsize{30}{40}\selectfont Samasya}}}
\date{%You can put a date here. \today works fine as well.
    }

% Footnote without marker

\newcommand\blfootnote[1]{%
  \begingroup
  \renewcommand\thefootnote{}\footnote{#1}%
  \addtocounter{footnote}{-1}%
  \endgroup
}


\begin{document}

\maketitle

Samasya is a mathematics discussion and problem solving club.
We discuss a variety of mathematical topics and solve problems as well.
We encourage participants to have a look at these problems%\footnote{The problems are not necessarily in the order of difficulty}
before the meeting.
Discussion, however, will not be limited to these problems.
Participants can bring their own problems or mathematical ideas they wish to discuss.\\
\hrule

\textbf{Date: 2\textsuperscript{nd} October, 2015}%Date for the meeting
\\
\textbf{Time: 9:00 p.m.}%Time for the meeting
\\
\textbf{Venue: OPB LAN Room}%Venue for the meeting
\\
\hrule

%Insert problems in the environment prob.
%\begin{prob}
%insert problem statement here
%\end{prob}

\begin{prob}
Let $sd$ be a function from $\mathbb{N}$ to $\mathbb{N}$ such that $sd(x)$ is the sum of digits of $x$ when written in base $10$. What is the value of $sd(sd(sd(4444^{4444})))$?
\end{prob}

\begin{prob}
Let $\{a_n\}$ and $\{b_n\}$ be two sequences of real numbers. Also, suppose that $\{a_n\}$ is a subsequence of $\{b_n\}$ and $\{b_n\}$ is a subsequence of $\{a_n\}$. Do there exists such sequences if $\{a_n\} \neq \{b_n\}$? What if $\{a_n\}$ does not converge? But if $\{a_n\}$ does converge, and there exists a sequence $\{b_n\}$ with the mentioned property of being a subsequence of $\{a_n\}$ and vice versa, is it true that the two sequences must be the same? 
\end{prob}

\begin{prob}
A function $f$ from $\mathbb{R}$ to $\mathbb{R}$ is said to have the intermediate value property if for all $a$, $b \in \mathbb{R}$ such that $a<b$, and for all $c$ between $f(a)$ and $f(b)$, there exists an $x \in (a,b)$ such that $f(x)=c$. It turns out that even if a function has the intermediate value property, it need not be continuous. To show such a function is continuous, one must put additional constraints on it. It turns out that if the set $S_r = \{x\ :\ f(x)=r\}$ is closed for all rational numbers $r$, then the function $f$ does turn out to be continuous. Prove it.
\end{prob}

\blfootnote{The past problems and solutions are available at \href{http://samasya.github.io}{samasya.github.io}.}
\end{document}
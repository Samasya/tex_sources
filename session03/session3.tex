\documentclass[10pt,a4paper,notitlepage]{article}
\usepackage[utf8]{inputenc}
\usepackage{amsmath}
\usepackage{amsfonts}
\usepackage{amssymb}
\usepackage{amsthm}
\usepackage{graphicx}
\usepackage[left=2cm,right=2cm,top=2cm,bottom=2cm]{geometry}
\usepackage{anyfontsize}
\usepackage{parskip}

%Theorem environment for problem
\theoremstyle{definition}
\newtheorem{prob}{Problem}

%Symbol for \divides
\newcommand{\divides}{\ensuremath{\nobreak \, \vert\, \nobreak}}

\makeatletter
%Harmonise parskip and theorem environments
\def\thm@space@setup
{
    \thm@preskip=\parskip
    \thm@postskip=0pt
}

%Different title style for Samasya
\def\@maketitle{%
    \newpage
    \null
    \vskip 2em%
    \begin{center}%
        \let \footnote \thanks
        {\LARGE \@title \par}%
        \vskip 1.5em%
        %        {\large
        %            \lineskip .5em%
        %            \begin{tabular}[t]{c}%
        %                \@author
        %            \end{tabular}\par}
        %        \vskip 1em%
        {\large \@date}%
    \end{center}
    %    \par
    %    \vskip 1.5em}
}
\makeatother

\title{\textrm{\textbf{\fontsize{30}{40}\selectfont Samasya}}}
\date{%You can put a date here. \today works fine as well.
    }

\begin{document}

\maketitle

Samasya is a mathematics discussion and problem solving club.
We discuss a variety of mathematical topics and solve problems as well.
We encourage participants to have a look at these problems%\footnote{The problems are not necessarily in the order of difficulty}
before the meeting.
Discussion, however, will not be limited to these problems.
Participants can bring their own problems or mathematical ideas they wish to discuss.\\
\hrule

\textbf{Date: 21\textsuperscript{st} August, 2015}%Date for the meeting
\\
\textbf{Time: 9:00 p.m.}%Time for the meeting
\\
\textbf{Venue: O.P.B. WiFi Room}%Venue for the meeting
\\
\hrule

%Insert problems in the environment prob.
%\begin{prob}
%insert problem statement here
%\end{prob}

\begin{prob}
Define a subset $A$ of $\mathbb{N}$ to be \emph{measurable} if the sequence $\{a_n\}$ converges, where $a_n = \frac{\#(A \cap \{1,2,\ldots,n\})}{n}$. Show that the intersection of two measurable subsets need not be measurable.
\end{prob}

\begin{prob}
Consider the sequence $\{s_n\}$, where,
$$s_k = \sum_{i=0}^{k}\frac{1}{i!}$$
Show that this sequence converges, and it converges to an irrational number.
\end{prob}

\begin{prob}
Prove that every positive real number $b$ has a square root by showing that the sequence $\{a_n\}$ with $a_{1} = 1$ and $a_{k+1} = \frac{a_k + \frac{b}{a_k}}{2}$ converges to a number $x$ such that $x^2=b$.
\end{prob}

\end{document}
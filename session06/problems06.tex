\documentclass[10pt,a4paper,notitlepage]{article}
\usepackage[utf8]{inputenc}
\usepackage{amsmath}
\usepackage{amsfonts}
\usepackage{amssymb}
\usepackage{amsthm}
\usepackage{graphicx}
\usepackage[left=2cm,right=2cm,top=2cm,bottom=2cm]{geometry}
\usepackage{anyfontsize}
\usepackage{parskip}

%Theorem environment for problem
\theoremstyle{definition}
\newtheorem{prob}{Problem}

%Symbol for \divides
\newcommand{\divides}{\ensuremath{\nobreak \, \vert\, \nobreak}}

\makeatletter
%Harmonise parskip and theorem environments
\def\thm@space@setup
{
    \thm@preskip=\parskip
    \thm@postskip=0pt
}

%Different title style for Samasya
\def\@maketitle{%
    \newpage
    \null
    \vskip 2em%
    \begin{center}%
        \let \footnote \thanks
        {\LARGE \@title \par}%
        \vskip 1.5em%
        %        {\large
        %            \lineskip .5em%
        %            \begin{tabular}[t]{c}%
        %                \@author
        %            \end{tabular}\par}
        %        \vskip 1em%
        {\large \@date}%
    \end{center}
    %    \par
    %    \vskip 1.5em}
}
\makeatother

\title{\textrm{\textbf{\fontsize{30}{40}\selectfont Samasya}}}
\date{%You can put a date here. \today works fine as well.
    }

\begin{document}

\maketitle

Samasya is a mathematics discussion and problem solving club.
We discuss a variety of mathematical topics and solve problems as well.
We encourage participants to have a look at these problems%\footnote{The problems are not necessarily in the order of difficulty}
before the meeting.
Discussion, however, will not be limited to these problems.
Participants can bring their own problems or mathematical ideas they wish to discuss.\\
\hrule

\textbf{Date: Friday, 11th September}%Date for the meeting
\\
\textbf{Time: 9:00 p.m.}%Time for the meeting
\\
\textbf{Venue: OPB LAN Room}%Venue for the meeting
\\
\hrule

%Insert problems in the environment prob.
%\begin{prob}
%insert problem statement here
%\end{prob}

\begin{prob}
A set $K$ of points with integer coordinates in $\mathbb{R}^2$ is said to be connected if for every pair of
points $a,\ b \in K$, there exists a finite sequence (of length $m$) of points $\{a_i\}$ such that $a_1=a$, $a_m=b$ and $|a_{k+1}-a_k|=1$ for $1\leq k < m$. Let $\bigtriangleup K = \{a-b\ |\ a,\ b \in K$,\ $a \neq b\}$. What is the maximum value of $\bigtriangleup K$ when $K$ varies over all connected sets of size $2n+1$, where $n \in \mathbb{N}$.
\end{prob}

\begin{prob}
Let $ABC$ be a triangle, and let $D$ be the point where the incircle meets $BC$. Let $J_b$ and $J_c$ be the incentres of the triangles $ABD$ and $ACD$ respectively. Prove that the circumcentre of the triangle $AJ_bJ_c$ lies on the angle bisector of $\angle BAC$
\end{prob}

\begin{prob}
If $a_1, a_2, \ldots, a_n$ are real numbers, prove that:
$$\sum_{i=1}^n a_i^2 - \sum_{i=1}^n a_ia_{i+1} \leq \left\lfloor \frac{n}{2} \right\rfloor (M-m)^2$$
where $a_{n+1}=a_1$, $M = \max(\{a_i\})$ and $m=\min(\{a_i\})$.
\end{prob}

\begin{prob}
Find all polynomials $p$ with integer coefficients such that the set $S_p = \{p(a)\ |\ a \in \mathbb{Z}\}$ has a geometric progression.
\end{prob}

\begin{prob}
Consider a grid of $n^2$ dots where $n \in 2\mathbb{Z}$ through which a closed path goes through. Prove that there exists a pair of adjacent vertices such that if those two vertices are deleted, the part breaks into two disconnected parts, such that each of their lengths is atleast a quarter of the length of the original path.
\end{prob}

\end{document}
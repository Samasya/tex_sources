\documentclass[10pt,a4paper,notitlepage]{article}
\usepackage[utf8]{inputenc}
\usepackage{amsmath}
\usepackage{amsfonts}
\usepackage{amssymb}
\usepackage{amsthm}
\usepackage{graphicx}
\usepackage[left=2cm,right=2cm,top=2cm,bottom=2cm]{geometry}
\usepackage{anyfontsize}
\usepackage{parskip}
\usepackage{hyperref}

%Theorem environment for problem
\theoremstyle{definition}
\newtheorem{prob}{Problem}

%Symbol for \divides
\newcommand{\divides}{\ensuremath{\nobreak \, \vert\, \nobreak}}

\makeatletter
%Harmonise parskip and theorem environments
\def\thm@space@setup
{
    \thm@preskip=\parskip
    \thm@postskip=0pt
}

%Different title style for Samasya
\def\@maketitle{%
    \newpage
    \null
    \vskip 2em%
    \begin{center}%
        \let \footnote \thanks
        {\LARGE \@title \par}%
        \vskip 1.5em%
        %        {\large
        %            \lineskip .5em%
        %            \begin{tabular}[t]{c}%
        %                \@author
        %            \end{tabular}\par}
        %        \vskip 1em%
        {\large \@date}%
    \end{center}
    %    \par
    %    \vskip 1.5em}
}
\makeatother

\title{\textrm{\textbf{\fontsize{30}{40}\selectfont Samasya}}}
\date{%You can put a date here. \today works fine as well.
    }

% Footnote without marker

\newcommand\blfootnote[1]{%
  \begingroup
  \renewcommand\thefootnote{}\footnote{#1}%
  \addtocounter{footnote}{-1}%
  \endgroup
}


\begin{document}

\maketitle

Samasya is a mathematics discussion and problem solving club.
We discuss a variety of mathematical topics and solve problems as well.
We encourage participants to have a look at these problems%\footnote{The problems are not necessarily in the order of difficulty}
before the meeting.
Discussion, however, will not be limited to these problems.
Participants can bring their own problems or mathematical ideas they wish to discuss.\\
\hrule

\textbf{Date: 6\textsuperscript{th} November, 2015}%Date for the meeting
\\
\textbf{Time: 9:00 p.m.}%Time for the meeting
\\
\textbf{Venue: OPB LAN Room}%Venue for the meeting
\\
\hrule

%Insert problems in the environment prob.
%\begin{prob}
%insert problem statement here
%\end{prob}

\begin{prob}
Given a balance scale, and $N$ coins, such that $N-1$ have the same weight, but one is either heavier or lighter (you do not know which), what is the minimum number of weighings $W$ you need to make to determine the odd coin? Conversely, given that you can weigh at most $W$ times, what is the maximum number of coins $N$ from which you can pick out the faulty coin?
\end{prob}

\begin{prob}
For which integers $n>2$ does the set of positive integers less than and relatively prime to $n$ and greater than $1$ constitute an arithmetic progression?
\end{prob}

\begin{prob}
Given a positive natural number $n$, a partition of $n$ is a non-decreasing strictly positive sequence of integers $\{a_1, a_2, \ldots a_m\}$ such that $\sum_{i=1}^{m}a_i = n$. Given a particular partition of $P$, let $A(P)$ be the number of ones that appear in $P$, and let $B(P)$ be the number of distinct elements in the partition. For example, given the partition $P = \{1,1,1,1\}$ of $4$, $A(P)=4$ and $B(P) = 1$. Let the set $S = \{P_1, P_2, \ldots P_k\}$ be all the partitions of a given $n$. Show that
$$\sum_{i=1}^{k} A(P_i) = \sum_{i=1}^{k} B(P_i)$$
\end{prob}

\begin{prob}
If $a$ and $b$ are two positive irrational numbers such that $\frac{1}{a} + \frac{1}{b} = 1$, then show that the sequence $\{a, b, 2a, 2b, 3a, 3b, \ldots\}$ contains every natural number exactly once.
\end{prob}

\begin{prob}[Group Theory]
Let $G$ be a finite abelian group. It is known that for all positive $n$, the number of solutions to $x^n=e$ is at most $n$, where $e$ is the identity element of the group. Show that the group is cyclic, i.e. there exists an element $a$ in the group such that every element of the group is of the form $a^m$, where $m$ is an integer.
\end{prob}

\blfootnote{The past problems and solutions are available at \href{http://samasya.github.io}{samasya.github.io}.}
\end{document}
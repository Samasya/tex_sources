\documentclass[10pt,a4paper,notitlepage]{article}
\usepackage[utf8]{inputenc}
\usepackage{amsmath}
\usepackage{amsfonts}
\usepackage{amssymb}
\usepackage{amsthm}
\usepackage{graphicx}
\usepackage[left=2cm,right=2cm,top=2cm,bottom=2cm]{geometry}
\usepackage{anyfontsize}
\usepackage{parskip}
\usepackage{hyperref}

%Theorem environment for problem
\theoremstyle{definition}
\newtheorem{prob}{Problem}

%Symbol for \divides
\newcommand{\divides}{\ensuremath{\nobreak \, \vert\, \nobreak}}

\makeatletter
%Harmonise parskip and theorem environments
\def\thm@space@setup
{
    \thm@preskip=\parskip
    \thm@postskip=0pt
}

%Different title style for Samasya
\def\@maketitle{%
    \newpage
    \null
    \vskip 2em%
    \begin{center}%
        \let \footnote \thanks
        {\LARGE \@title \par}%
        \vskip 1.5em%
        %        {\large
        %            \lineskip .5em%
        %            \begin{tabular}[t]{c}%
        %                \@author
        %            \end{tabular}\par}
        %        \vskip 1em%
        {\large \@date}%
    \end{center}
    %    \par
    %    \vskip 1.5em}
}
\makeatother

\title{\textrm{\textbf{\fontsize{30}{40}\selectfont Samasya}}}
\date{%You can put a date here. \today works fine as well.
    }

% Footnote without marker

\newcommand\blfootnote[1]{%
  \begingroup
  \renewcommand\thefootnote{}\footnote{#1}%
  \addtocounter{footnote}{-1}%
  \endgroup
}


\begin{document}

\maketitle

Samasya is a mathematics discussion and problem solving club.
We discuss a variety of mathematical topics and solve problems as well.
We encourage participants to have a look at these problems%\footnote{The problems are not necessarily in the order of difficulty}
before the meeting.
Discussion, however, will not be limited to these problems.
Participants can bring their own problems or mathematical ideas they wish to discuss.\\
\hrule

\textbf{Date: 18\textsuperscript{th} September, 2015}%Date for the meeting
\\
\textbf{Time: 9:00 p.m.}%Time for the meeting
\\
\textbf{Venue: OPB LAN Room}%Venue for the meeting
\\
\hrule

%Insert problems in the environment prob.
%\begin{prob}
%insert problem statement here
%\end{prob}

\begin{prob}
Does there exist a function $f$ from $\mathbb{R}$ to $\mathbb{R}$ such that for any non-empty open interval $(a,b) \subset \mathbb{R}$, the restriction of $f$ to $(a,b)$ is onto on $\mathbb{R}$, i.e. $\forall y \in \mathbb{R}$ there exists an $x \in (a,b)$ such that $f(x)=y$. 
\end{prob}

\begin{prob}
It is a known fact that the set of rationals $\mathbb{Q}$ is countable. Hence, one can enumerate the rationals in a sequence, such that every rational number appears in the sequence exactly once. Does there exist an enumeration of the rationals in a sequence $q_1, q_2, \ldots$ such that $\sum (q_i-q_{i+1})^2$ converges?
\end{prob}

\begin{prob}
Form a “triangle” with 10 blocks in its top row, 9 blocks in
the next row, etc., until the bottom row has one block. Each
row is centered below the row above it. Color the blocks in
the top row red, white and blue in any way. Use these two
rules to color the remaining rows of the triangle:
\begin{itemize}
\item If two consecutive blocks have the same color, the block
between them in the row below also has the same color.
\item If two consecutive blocks have different colors, the block
between them in the row below has the third color.
\end{itemize}
Is there a nice way of directly computing the color of the block
in the bottom row, given the top row, without actually computing the
colors of the rows in between?
\end{prob}

\begin{prob}
Given a polygon of area $A$, show that one can, using finitely many straight edge cuts and joins, cut up the polygon and reassemble it into any other polygon of area $A$.
\end{prob}

\begin{prob}
Most people believe the derivatives of differentiable functions are continuous without ever having seen the proof for it, or having proven it themselves. It's actually not true. Construct an example to show the fact. Furthermore, show that even though the derivative may not be continuous, it still has the intermediate value property, i.e. if $f'(x)=a$ and $f'(y)=b$, then $\forall c \in (a,b)$, there exists a $z \in (x,y)$ such that $f(z)=c$.
\end{prob}

\begin{prob}
There are $432$ equidistant points placed on a circle, with each point having one of four colours (red, blue, green and yellow), and there are $108$ points of each color. Show that there exist four congruent triangles, one with all red points, one with all blue points, one with all green points and one with all yellow points.
\end{prob}

\blfootnote{The past problems and solutions are available at \href{http://samasya.github.io}{samasya.github.io}.}
\end{document}
\documentclass[10pt,a4paper,notitlepage]{article}
\usepackage[utf8]{inputenc}
\usepackage{amsmath}
\usepackage{amsfonts}
\usepackage{amssymb}
\usepackage{amsthm}
\usepackage{graphicx}
\usepackage[left=2cm,right=2cm,top=2cm,bottom=2cm]{geometry}
\usepackage{anyfontsize}
\usepackage{parskip}
\usepackage{hyperref}

%Theorem environment for problem
\theoremstyle{definition}
\newtheorem{prob}{Problem}

%Symbol for \divides
\newcommand{\divides}{\ensuremath{\nobreak \, \vert\, \nobreak}}

\makeatletter
%Harmonise parskip and theorem environments
\def\thm@space@setup
{
    \thm@preskip=\parskip
    \thm@postskip=0pt
}

%Different title style for Samasya
\def\@maketitle{%
    \newpage
    \null
    \vskip 2em%
    \begin{center}%
        \let \footnote \thanks
        {\LARGE \@title \par}%
        \vskip 1.5em%
        %        {\large
        %            \lineskip .5em%
        %            \begin{tabular}[t]{c}%
        %                \@author
        %            \end{tabular}\par}
        %        \vskip 1em%
        {\large \@date}%
    \end{center}
    %    \par
    %    \vskip 1.5em}
}
\makeatother

\title{\textrm{\textbf{\fontsize{30}{40}\selectfont Samasya}}}
\date{%You can put a date here. \today works fine as well.
    }

% Footnote without marker

\newcommand\blfootnote[1]{%
  \begingroup
  \renewcommand\thefootnote{}\footnote{#1}%
  \addtocounter{footnote}{-1}%
  \endgroup
}


\begin{document}

\maketitle

Samasya is a mathematics discussion and problem solving club.
We discuss a variety of mathematical topics and solve problems as well.
We encourage participants to have a look at these problems%\footnote{The problems are not necessarily in the order of difficulty}
before the meeting.
Discussion, however, will not be limited to these problems.
Participants can bring their own problems or mathematical ideas they wish to discuss.\\
\hrule

\textbf{Date: 16\textsuperscript{th} October, 2015}%Date for the meeting
\\
\textbf{Time: 9:00 p.m.}%Time for the meeting
\\
\textbf{Venue: OPB LAN Room}%Venue for the meeting
\\
\hrule

%Insert problems in the environment prob.
%\begin{prob}
%insert problem statement here
%\end{prob}

\begin{prob}
Let $S$ be a subset of the power set of $\mathbb{N}$ such that for any $A$, $B \in S$, either $A \subset B$ or $B \subset A$. Can $S$ be an uncountable set?
\end{prob}

\begin{prob}
Let $d \leq n$, where $d$ and $n$ are positive integers and $d$ is even. How many subsets of $\{1, 2, \ldots, n\}$ exist such that any two have symmetric difference of size at most $d$? Symmetric difference of two sets $A$ and $B$ is $(A \setminus B) \cup (B \setminus A)$.
\end{prob}

\begin{prob}
An ordered field $\mathbb{F}$ is said to have the intermediate value property if for every continuous function $f:\ (a,b) \mapsto \mathbb{F}$, $f(a)<0$ and $f(b)>0$, there exists a $c \in (a,b)$, such that $f(c)=0$. Show that an ordered field with the intermediate value property has the least upper bound property.
\end{prob}

\begin{prob}
Let $1, n_1, n_2, \ldots$ be an increasing sequence of integers such that none of them are prime and any two distinct elements are co-prime. Show that the sum $\sum_{j=1}^\infty \frac{1}{n_j}$ converges. 
\end{prob}

\begin{prob}
Show that there are arbitrarily large $n$ for which $n^4+1$ has a prime divisor larger than $2n$.
\end{prob}

\blfootnote{The past problems and solutions are available at \href{http://samasya.github.io}{samasya.github.io}.}
\end{document}